\documentclass[12pt,letterpaper]{article}
\usepackage{graphicx,textcomp}
\usepackage{natbib}
\usepackage{setspace}
\usepackage{fullpage}
\usepackage{color}
\usepackage[reqno]{amsmath}
\usepackage{amsthm}
\usepackage{fancyvrb}
\usepackage{amssymb,enumerate}
\usepackage[all]{xy}
\usepackage{endnotes}
\usepackage{lscape}
\newtheorem{com}{Comment}
\usepackage{float}
\usepackage{hyperref}
\newtheorem{lem} {Lemma}
\newtheorem{prop}{Proposition}
\newtheorem{thm}{Theorem}
\newtheorem{defn}{Definition}
\newtheorem{cor}{Corollary}
\newtheorem{obs}{Observation}
\usepackage[compact]{titlesec}
\usepackage{dcolumn}
\usepackage{tikz}
\usetikzlibrary{arrows}
\usepackage{multirow}
\usepackage{xcolor}
\newcolumntype{.}{D{.}{.}{-1}}
\newcolumntype{d}[1]{D{.}{.}{#1}}
\definecolor{light-gray}{gray}{0.65}
\usepackage{url}
\usepackage{listings}
\usepackage{color}

\definecolor{codegreen}{rgb}{0,0.6,0}
\definecolor{codegray}{rgb}{0.5,0.5,0.5}
\definecolor{codepurple}{rgb}{0.58,0,0.82}
\definecolor{backcolour}{rgb}{0.95,0.95,0.92}

\lstdefinestyle{mystyle}{
	backgroundcolor=\color{backcolour},   
	commentstyle=\color{codegreen},
	keywordstyle=\color{magenta},
	numberstyle=\tiny\color{codegray},
	stringstyle=\color{codepurple},
	basicstyle=\footnotesize,
	breakatwhitespace=false,         
	breaklines=true,                 
	captionpos=b,                    
	keepspaces=true,                 
	numbers=left,                    
	numbersep=5pt,                  
	showspaces=false,                
	showstringspaces=false,
	showtabs=false,                  
	tabsize=2
}
\lstset{style=mystyle}
\newcommand{\Sref}[1]{Section~\ref{#1}}
\newtheorem{hyp}{Hypothesis}

\title{Problem Set 3}
\date{Due: March 26, 2023}
\author{Cianna Devitt (17321885)}


\begin{document}
	\maketitle
	\section*{Instructions}
	\begin{itemize}
	\item Please show your work! You may lose points by simply writing in the answer. If the problem requires you to execute commands in \texttt{R}, please include the code you used to get your answers. Please also include the \texttt{.R} file that contains your code. If you are not sure if work needs to be shown for a particular problem, please ask.
\item Your homework should be submitted electronically on GitHub in \texttt{.pdf} form.
\item This problem set is due before 23:59 on Sunday March 26, 2023. No late assignments will be accepted.
	\end{itemize}

	\vspace{.25cm}
\section*{Question 1}
\vspace{.25cm}
\noindent We are interested in how governments' management of public resources impacts economic prosperity. Our data come from \href{https://www.researchgate.net/profile/Adam_Przeworski/publication/240357392_Classifying_Political_Regimes/links/0deec532194849aefa000000/Classifying-Political-Regimes.pdf}{Alvarez, Cheibub, Limongi, and Przeworski (1996)} and is labelled \texttt{gdpChange.csv} on GitHub. The dataset covers 135 countries observed between 1950 or the year of independence or the first year forwhich data on economic growth are available ("entry year"), and 1990 or the last year for which data on economic growth are available ("exit year"). The unit of analysis is a particular country during a particular year, for a total $>$ 3,500 observations. 

\begin{itemize}
	\item
	Response variable: 
	\begin{itemize}
		\item \texttt{GDPWdiff}: Difference in GDP between year $t$ and $t-1$. Possible categories include: "positive", "negative", or "no change"
	\end{itemize}
	\item
	Explanatory variables: 
	\begin{itemize}
		\item
		\texttt{REG}: 1=Democracy; 0=Non-Democracy
		\item
		\texttt{OIL}: 1=if the average ratio of fuel exports to total exports in 1984-86 exceeded 50\%; 0= otherwise
	\end{itemize}
	
\end{itemize}
\newpage
\noindent Please answer the following questions:

\begin{enumerate}
	\item Construct and interpret an unordered multinomial logit with \texttt{GDPWdiff} as the output and "no change" as the reference category, including the estimated cutoff points and coefficients.
	\newline\texttt{Load in data and set factors, releveled with 'no change' as reference category}
	\begin{lstlisting}[language=R]
		gdpChange$GDPWdiff <- factor(ifelse(sign(gdpChange$GDPWdiff) == -1, 'negative',
		ifelse(sign(gdpChange$GDPWdiff) == 0, 'no change','positive')))
		
		
		gdpChange$GDPWdiff <- relevel(gdpChange$GDPWdiff, ref = 'no change')
	\end{lstlisting}
\newline\texttt{Running base multinomial logit unordered}
  \begin{lstlisting}[language=R]
	multinom_model1 <- multinom(GDPWdiff ~ REG + OIL,
	data = gdpChange)
	
	summary(multinom_model1)
	
	
	Coefficients:
	(Intercept)      REG      OIL
	negative    3.805370 1.379282 4.783968
	positive    4.533759 1.769007 4.576321
	
	Std. Errors:
	(Intercept)       REG      OIL
	negative   0.2706832 0.7686958 6.885366
	positive   0.2692006 0.7670366 6.885097
	
	Residual Deviance: 4678.77 
	AIC: 4690.77 
\end{lstlisting}
\newline\texttt{In a given country, there is an increase baseline odds of 1.76 that the difference in GDP will be positive. }
	
	\item Construct and interpret an ordered multinomial logit with \texttt{GDPWdiff} as the outcome variable, including the estimated cutoff points and coefficients.
	\newline\texttt{Running base multinomial logit ordered}
	\begin{lstlisting}[language=R]
		multinom_model2 <- polr(GDPWdiff ~ REG + OIL,
		data = gdpChange)
		
		summary(multinom_model2)
		
		
		Coefficients:
		Value Std. Error t value
		REG  0.4102    0.07518   5.456
		OIL -0.1788    0.11546  -1.549
		
		Intercepts:
		Value    Std. Error t value 
		no change|negative  -5.3199   0.2523   -21.0878
		negative|positive   -0.7036   0.0476   -14.7933
		
		Residual Deviance: 4686.606 
		AIC: 4694.606
	\end{lstlisting}

\newline\texttt{Finding proportional odds ratios}
	\begin{lstlisting}[language=R]
		ci <- confint(multinom_model2)
		confint.default(multinom_model2)
		exp(cbind(OR= coef(multinom_model2), ci))
		
		
		  OR     2.5 %   97.5 %
		REG 1.5070726 1.3012858 1.747374
		OIL 0.8362455 0.6680959 1.050857
	\end{lstlisting}

\newline\texttt{For a unit increase in democratic status (REG), the odds of GDP difference increase
	by 1.5, holding constant all other variables
	 For a unit increase in ratio of fuel exports (OIL),
	 the odds of being more likely positive GDP difference is 
	 0.836, holding all other variables constant.}
	
\end{enumerate}

\section*{Question 2} 
\vspace{.25cm}

\noindent Consider the data set \texttt{MexicoMuniData.csv}, which includes municipal-level information from Mexico. The outcome of interest is the number of times the winning PAN presidential candidate in 2006 (\texttt{PAN.visits.06}) visited a district leading up to the 2009 federal elections, which is a count. Our main predictor of interest is whether the district was highly contested, or whether it was not (the PAN or their opponents have electoral security) in the previous federal elections during 2000 (\texttt{competitive.district}), which is binary (1=close/swing district, 0="safe seat"). We also include \texttt{marginality.06} (a measure of poverty) and \texttt{PAN.governor.06} (a dummy for whether the state has a PAN-affiliated governor) as additional control variables. 

\begin{enumerate}
	\item [(a)]
	Run a Poisson regression because the outcome is a count variable. Is there evidence that PAN presidential candidates visit swing districts more? Provide a test statistic and p-value.
	
	\newline\texttt{Running Poisson Regression Model}
	\begin{lstlisting}[language=R]
		Mex_poisson <- glm(PAN.visits.06 ~ competitive.district + marginality.06
		+ PAN.governor.06, data = MexicoMuniData, family = poisson)
		
		summary(Mex_poisson)
		
		
		Coefficients:
		Estimate Std. Error z value
		(Intercept)          -3.81023    0.22209 -17.156
		competitive.district -0.08135    0.17069  -0.477
		marginality.06       -2.08014    0.11734 -17.728
		PAN.governor.06      -0.31158    0.16673  -1.869
		Pr(>|z|)    
		(Intercept)            <2e-16 ***
		competitive.district   0.6336    
		marginality.06         <2e-16 ***
		PAN.governor.06        0.0617 .  
		---
		Signif. codes:  
		0 ‘***’ 0.001 ‘**’ 0.01 ‘*’ 0.05 ‘.’ 0.1 ‘ ’ 1
		
		(Dispersion parameter for poisson family taken to be 1)
		
		Null deviance: 1473.87  on 2406  degrees of freedom
		Residual deviance:  991.25  on 2403  degrees of freedom
		AIC: 1299.2
		
		Number of Fisher Scoring iterations: 7
	\end{lstlisting}

	\item [(b)]
	\texttt{With a unit increase in 'marginality.06' ie a measure of poverty, had a diminished liklihood by 2.08 of having candidate visitations, holding all other variables constant. 
		With a unit increase in 'PAN.governor.06' ie whetther state has a PAN=affiliated governor, had a diminished liklihood by 0.31 of having candidate visitations, holding all other variables constant.}
	\item [(c)]
	Provide the estimated mean number of visits from the winning PAN presidential candidate for a hypothetical district that was competitive (\texttt{competitive.district}=1), had an average poverty level (\texttt{marginality.06} = 0), and a PAN governor (\texttt{PAN.governor.06}=1).
	
	
	\vspace{2cm}
	
	
	\texttt{Getting Fitted Values}
	\newline\texttt{$ \lambda_i = e^\beta_o + \beta_1_X_i$}
	
	\begin{lstlisting}[language=R]
		lambda30 <- exp(coeffs[1] + coeffs[2]*1)
		lambda30
		
		
		(Intercept) 
		0.02041293
	\end{lstlisting}
\end{enumerate}

\end{document}
